\documentclass{article}
 
\begin{document}
\begin{enumerate} 

    \item OpenVPN:\\

    Connettersi tramite OpenVPN ed esplorare l’interfaccia di gestione del proprio
    server (10.19.0.101-112) (user: root password: superuser) e documentare
    azioni e informazioni disponibili.
    (Gli account OpenVPN sono gestiti dal docente, che distribuisce ad ogni gruppo
    un certificato personalizzato da importare nel proprio OpenVPN client – esempio
    di nome di un certificato: SystemManagement-udp-1194-grp5.ovpn e
    SystemManagement-udp-1194-grp5.p12).\\

   Installazione OpenVPN - Ubuntu x64:\\

    \begin{enumerate}
        \item apt-get install openvpn \\ 
        Installiamo openvpn sulla nostra macchina
        \item openvpn --version \\
        Verifichiamo che l'installazione sia andata a buon fine
        \item openvpn --config client.ovpn \\
        Ci posizioniamo nella cartella (unzippata) che abbiamo scaricato da ICorsi,
        avviamo il client con il certificato corretto (.ovpn) che ci è stato fornito.\\

    \end{enumerate}

    Problematiche: \\
    Il server mette a disposizione una finestra con la quale è possibile lavorare in modalità
    grafica stile desktop remoto. Tale applicazione fa scaricare un file .JNLP dal server
    (lanciando il controllo renoto dalla console di gestione). Qualora il certificato di
    sicurezza fosse scaduto, si dovrà procedere a creare il trust al server nella macchina
    client (che altrimenti ne blocca l’esecuzione): aggiungere l’IP del server nelle Exception
    Site List di Java. \\
    Utilizzando OpenJDK non è possibile sfruttare l'interfaccia grafica di gestione, per questo 
    motivo abbiamo dovuto installare Oracle JRE.

    Installazione OpenVPN - Windows 10 x64:\\


    \item Installazione di VMWare: \\
    Dopo esserci collegati all'interfaccia di gestione della macchina, sulla scheda abbiamo
    Remot Control abbiamo scaricato il file .JNLP ed attraverso la sua esecuzione
    abbiamo installato l'ipervisor: abbiamo settato un Ip statico pubblico in modo da non dover
    accedere ogni volta tramite VPN. \\
    Abbiamo utilizzato per installare VMWare una chiavete collegata direttamente al server.
    
    \item \dots 
    \end{enumerate}
\end{document}
